\documentclass[a4paper,10pt]{article}

\usepackage[french]{babel}
\usepackage[T1]{fontenc}
\usepackage[utf8]{inputenc}
\usepackage{amsmath}
\usepackage{graphicx}
\usepackage[colorinlistoftodos,prependcaption,textsize=tiny]{todonotes}
\usepackage{eurosym}
\usepackage[toc,page]{appendix}

%3 lignes suivantes : Remet le compteur de section à 0 à chaque nouvelle partie
\makeatletter
\@addtoreset{section}{part}
\makeatother

\title{Analyse décisionnelle pour l'entreprise FaBrique}
\author{Heptanôme 4203}
\date{\today}

\begin{document}
\maketitle 

%%%%%%%%%%%%%%%%%%%%%%%%%%%%%%%%
%%%%%%%% Introduction %%%%%%%%%%
%%%%%%%%%%%%%%%%%%%%%%%%%%%%%%%%

\section*{Introduction}

  L'entreprise FaBrique a émis une demande d'analyse décisionnelle concernant la mise en place d'un plan de fabrication hebdomadaire optimal. L'entreprise fabrique différentes gammes de produits grâce à l'utilisation de machines au sein de sa structure. La demande du client nous impose de respecter des contraintes logistiques au sein de l'entreprise, mais également d'analyser les besoins unitaires de chaque responsable. Dans un premier temps, nous chercherons une solution optimale répondant de manière individuelle au besoin de chaque responsable, tout en tenant compte des contraintes identifiées au préalable. De manière à fournir un unique plan optimal pour toute l'entreprise, avec consultation du responsable d'entreprise, nous réaliserons ensuite une analyse multicritère afin d'établir un plan permettant un compromis satisfaisant pour chacune des parties. Nous inclurons systématiquement l'analyse et la modélisation mathématique effectuées en amont de la résolution du problème.

%%%%%%%%%%%%%%%%%%%%%%%%%%%%%%%%%%%%%%%%
%%%%%% Partie PROG MONOCRITERE %%%%%%%%
%%%%%%%%%%%%%%%%%%%%%%%%%%%%%%%%%%%%%%%%
\newpage
\part{Programmation linéaire monocritère}

%%%%%%%%%%%%%%%%%%%%%%%%%%%%%%%%%%%%%%%%%%%%
%%%%%% Modélisation des contraintes %%%%%%%%
%%%%%%%%%%%%%%%%%%%%%%%%%%%%%%%%%%%%%%%%%%%%

\section{Modélisation des contraintes}
L'analyse des contraintes est un facteur clé dans la réalisation d'un nouveau plan de développement pour l'entreprise ainsi que la première étape de notre étude.
\vspace{\baselineskip}

Conformément aux besoins clients émis, nous veillerons dans un premier temps à prendre en compte \textbf{la fabrication d'au moins 5 unités de produit A et 5 unités de produit B par semaine}. Afin d'établir un modèle mathématique cohérent, \textbf{nous veillerons également à éliminer les cas où la quantité de produits C, D, E ou F serait négative}.\newline

Notons $n_X$ le nombre de produits de dénomination X à produire par l'intermédiaire du plan de fabrication. Lors de l'exécution du plan, nous veillerons donc à respecter les critères suivants :

$$n_A \geq 5$$
$$n_B \geq 5$$
$$n_C \geq 0$$
$$n_D \geq 0$$
$$n_E \geq 0$$
$$n_F \geq 0$$

Nous établirons de plus un plan de fabrication \textbf{respectant les contraintes de stockage des matières premières dans l'entreprise}.

De manière à respecter ces critères, nous ne dépasserons pas l'utilisation de :
\newline
\begin{itemize}
\item[\textbullet] 350 unités de matière 1
\item[\textbullet] 620 unités de matière 2
\item[\textbullet] 485 unités de matière 3\newline
\end{itemize}

Compte-tenu des quantités de matières premières nécessaires à la fabrication de chaque objet, nous modélisons cette contrainte de la manière suivante :
\newline
\begin{itemize}
\item[\textbullet]  \textbf{Matière 1 :} $n_A + 2 \times n_B + n_C + 5 \times n_D + 2 \times n_F \leq 350$
\item[\textbullet]  \textbf{Matière 2 :} $2 \times n_A + 2 \times n_B + n_C + 2 \times n_D + 2 \times n_E + n_F \leq 620$
\item[\textbullet]  \textbf{Matière 3 :} $n_A + 3 \times n_C + 2 \times n_D + 2 \times n_E \leq 485$\newline
\end{itemize}

Pour finir, nous veillerons à ne pas sur-solliciter les machines dans le cadre du plan de fabrication établi, de manière à ce qu'il puisse être exécuté dans un mode de fonctionnement des machines en 2 huit, 5 jours sur 7, soit 4800 minutes par semaine. Compte-tenu des temps unitaires d'usinage de chaque produit sur chaque machine, nous fournissons le modèle suivant :
\newline
\begin{itemize}
\item[\textbullet] \textbf{Machine 1 :} $8 \times n_A + 15 \times n_B + 5 \times n_D + 10 \times n_F \leq 4800$
\item[\textbullet] \textbf{Machine 2 :} $7 \times n_A + 12 \times n_B +2 \times n_C + 15 \times n_D + 7 \times n_E + 12 \times n_F \leq 4800$
\item[\textbullet] \textbf{Machine 3 :} $8 \times n_A + n_B + 11 \times n_C + 10 \times n_E + 25 \times n_F \leq 4800$
\item[\textbullet] \textbf{Machine 4 :} $2 \times n_A + 10 \times n_B + 5 \times n_C + 4 \times n_D + 13 \times n_E + 7 \times n_F \leq 4800$
\item[\textbullet] \textbf{Machine 5 :} $5 \times n_A + 8 \times n_C + 7 \times n_D + 10 \times n_E + 25 \times n_F \leq 4800$
\item[\textbullet] \textbf{Machine 6 :} $5 \times n_A + 5 \times n_B + 3 \times n_C + 12 \times n_D + 8 \times n_E + 6 \times n_F \leq 4800$
\item[\textbullet] \textbf{Machine 7 :} $5 \times n_A + 3 \times n_B +5 \times n_C +8 \times n_D + 7 \times n_F \leq 4800 $\newline
\end{itemize}


\fbox{
\begin{minipage}{1\textwidth}
     \begin{center}
     
     \textbf{\textit{Modèle récapitulatif des contraintes}}
			$$n_A \geq 5$$ 
			$$n_B \geq 5$$ 
			$$n_C \geq 0$$ 
			$$n_D \geq 0$$
			$$n_E \geq 0$$ 
			$$n_F \geq 0$$ 
			$$n_A + 2 \times n_B + n_C + 5 \times n_D + 2 \times n_F \leq 350$$
			$$2 \times n_A + 2 \times n_B + n_C + 2 \times n_D + 2 \times n_E + n_F \leq 620$$
			$$n_A + 3 \times n_C + 2 \times n_D + 2 \times n_E \leq 485$$
			$$8 \times n_A + 15 \times n_B + 5 \times n_D + 10 \times n_F \leq 4800$$
			$$7 \times n_A + 12 \times n_B +2 \times n_C + 15 \times n_D + 7 \times n_E + 12 \times n_F \leq 4800$$
			$$8 \times n_A + n_B + 11 \times n_C + 10 \times n_E + 25 \times n_F \leq 4800$$
			$$2 \times n_A + 10 \times n_B + 5 \times n_C + 4 \times n_D + 13 \times n_E + 7 \times n_F \leq 4800$$
			$$5 \times n_A + 8 \times n_C + 7 \times n_D + 10 \times n_E + 25 \times n_F \leq 4800$$
			$$5 \times n_A + 5 \times n_B + 3 \times n_C + 12 \times n_D + 8 \times n_E + 6 \times n_F \leq 4800$$
			$$5 \times n_A + 3 \times n_B +5 \times n_C +8 \times n_D + 7 \times n_F \leq 4800 $$
               
     \end{center}
\end{minipage}
}
\vspace{\baselineskip}

Après avoir analysé les contraintes relatives au fonctionnement logistique de l'entreprise, nous effectuons \textbf{une analyse des besoins émis par chacun des responsables} au sein de la structure.

%%%%%%%%%%%%%%%%%%%%%%%%%%%%%%%%%%%%%%%%%%
%%%%%% Modélisation des objectifs %%%%%%%%
%%%%%%%%%%%%%%%%%%%%%%%%%%%%%%%%%%%%%%%%%%

\section{Modélisation des objectifs}

%%%%%%%%%%%%%%%%%%%%%%%%%%%%%%%%%%%%%%%%%%%%
%%%%%%%%%%%%%%%% Comptable %%%%%%%%%%%%%%%%%
%%%%%%%%%%%%%%%%%%%%%%%%%%%%%%%%%%%%%%%%%%%%

\subsection{Comptable}

\emph{NB : Les calculs plus détaillés de cette section sont disponibles en annexe, page \pageref{annexe}.} \\

L'objectif principal du comptable est de réaliser le plus grand bénéfice possible, ceci en tenant compte non seulement des coûts de fonctionnement des machines mais également du coût d'achat des matières premières. De cette façon, nous déduisons la fonction $G_1$ à maximiser:

$$G_1(N) = 5,833 \times n_A + 11,617 \times n_B + 12,17 \times n_C + 1,3 \times n_D + 31,65 \times n_E + 27,48 \times n_F$$

Les coefficients devant chaque $n_X$, c'est-à-dire les bénéfices à la vente du produit X, ont été calculés compte tenu du coût des matières premières, du coût d'utilisation des machines et des bénéfices à la vente de chaque produit fini.\newline


La fonction $G$ à minimiser est donc :

$$G(N) = -5,833 \times n_A - 11,617 \times n_B - 12,17 \times n_C - 1,3 \times n_D - 31,65 \times n_E - 27,48 \times n_F$$

La solution de base trouvée en utilisant les contraintes de l'entreprise est la fabrication minimale de :\newline
\begin{itemize}
\item[\textbullet] 5 produits A
\item[\textbullet] 18 produits B
\item[\textbullet] 0 produit C
\item[\textbullet] 0 produit D
\item[\textbullet] 240 produits E
\item[\textbullet] 93 produits F\newline
\end{itemize}
Le bénéfice s'échelonne ainsi à environ + 10 410 unités.\newline

Cette solution semble cohérente et réaliste car les produits qui doivent être le plus produits sont les produits possédant la plus grande valeur ajoutée. Remarquons le bénéfice moyen d'environ 30 unités, très proche du bénéfice maximum fait sur un produit (le produit E) qui est de 31 unités.

%%%%%%%%%%%%%%%%%%%%%%%%%%%%%%%%%%%%%%%%%%%%
%%%%%%%%%% Responsable d'atelier %%%%%%%%%%%
%%%%%%%%%%%%%%%%%%%%%%%%%%%%%%%%%%%%%%%%%%%%

\subsection{Responsable d'atelier}

L'objectif du responsable d'atelier consiste à fabriquer une quantité maximale de produits. Cela se traduit par la fonction $F_1$ à maximiser :

$$F_1(N) = n_A + n_B + n_C + n_D + n_E + n_F$$.

d'où $F$ à minimiser suivante :
$$F(N) = - n_A - n_B - n_C - n_D - n_E - n_F$$

La solution de base trouvée en utilisant les contraintes de l'entreprise est la fabrication minimale de :\newline
\begin{itemize}
\item[\textbullet] 5 produits A
\item[\textbullet] 54 produits B
\item[\textbullet] 38 produit C
\item[\textbullet] 0 produit D
\item[\textbullet] 181 produits E
\item[\textbullet] 98 produits F\newline
\end{itemize}
Le nombre de produits réalisés serait donc de 376 produits au total, toutes gammes confondues.

%%%%%%%%%%%%%%%%%%%%%%%%%%%%%%%%%%%%%%%%%%%%
%%%%%%%%% Responsable des stocks %%%%%%%%%%%
%%%%%%%%%%%%%%%%%%%%%%%%%%%%%%%%%%%%%%%%%%%%

\subsection{Responsable des stocks}

Le but du responsable des stocks est de minimiser le nombre de produits et la quantité de matière première dans son stock. Cependant, en considérant cet objectif isolé, nous en arrivons de manière évidente à la conclusion que fermer l'entreprise serait l'option la plus adaptée. C'est pourquoi nous ajoutons une contrainte de réalisme, modélisant le fait que l'entreprise reste en activité.\newline

Nous avons calculé (lors des calculs pour le responsable d'atelier, pour maximiser la production) que la production maximale hebdomadaire de l'entreprise était de 376 produits. Afin de tendre vers un modèle cohérent, nous avons décidé que la production minimale pour que l'entreprise garde son activité, serait de 80\% de la production maximale, soit 300 produits par semaine.\newline

%%%%%%% LE CROUPIONG%%%%%%%%%%

Le "coût de stockage" d'un produit est égal à la quantité de matières premières utilisées pour la fabrication de ce produit, à laquelle on ajoute 1 (pour le produit lui-même). Nous obtenons donc la fonction $H$ à minimiser suivante :
$$ H(N)=  5n_A + 5n_B + 6n_C +10n_D + 5n_E + 4n_F $$

Pour satisfaire cet objectif, tout en respectant les contraintes de départ, nous devons suivre le plan de production hebdomadaire suivant :\newline
\begin{itemize}
\item[\textbullet] 5 produits A
\item[\textbullet] 32 produits B
\item[\textbullet] 0 produits C
\item[\textbullet] 0 produits D
\item[\textbullet] 122 produits E
\item[\textbullet] 140 produits F\newline
\end{itemize}

On arrive à un stockage optimal de 1355 unités.

Ce résultat est parfaitement cohérent car les produits E et F sont ceux qui nécessitent le moins de matières premières et donc qui occupent le moins de stock. Le produit F est le plus adapté mais il est plus long à produire et une production avec plus de produits F ne respecterait pas la contrainte de production minimale hebdomadaire.

%%%%%%%%%%%%%%%%%%%%%%%%%%%%%%%%%%%%%%%%%%%%
%%%%%%%%% Responsable commercial %%%%%%%%%%%
%%%%%%%%%%%%%%%%%%%%%%%%%%%%%%%%%%%%%%%%%%%%

\subsection{Responsable commercial}

Le but du responsable commercial est d'équilibrer les quantités de produits A et E. Avec ce seul objectif, il existe une infinité de solutions possibles, nous devons donc ajouter une contrainte.

Nous considérons une contrainte d'égalité entre les quantités de produits A et E, avec l'objectif de produire le plus de produits possibles. Nous allons donc, comme lors de l'étude des objectifs du responsable de l'atelier, minimiser la fonction :
$$F(N) = - n_A - n_B - n_C - n_D - n_E - n_F$$.
Cette fois, nous ajoutons une contrainte supplémentaire :
$$n_A = n_E$$
Nous obtenons le plan de production hebdomadaire suivant :
\begin{itemize}
\item[\textbullet] 119 produits A
\item[\textbullet] 6 produits B
\item[\textbullet] 42 produits C
\item[\textbullet] 0 produits D
\item[\textbullet] 119 produits E
\item[\textbullet] 87 produits F\newline
\end{itemize}

Nous avons ainsi bien autant de produits A que de E à la fin de la semaine, tout en conservant une production de 373 produits, soit 99\% du nombre de produits maximum produits sur une semaine.


%%%%%%%%%%%%%%%%%%%%%%%%%%%%%%%%%%%%%%%%%%%%%%
%%%%%%%%% Responsable du personnel %%%%%%%%%%%
%%%%%%%%%%%%%%%%%%%%%%%%%%%%%%%%%%%%%%%%%%%%%%

\subsection{Responsable du personnel}

L'objectif principal du responsable du personnel est de \textbf{limiter l'utilisation de la machine 4}, qui est très délicate. De cette façon, nous en déduisons la fonction $J$ à minimiser suivante :

$$J(N)=2 \times n_A + 10 \times n_B + 5 \times n_C + 4 \times n_D + 13 \times n_E + 7 \times n_F$$

Cette fonction correspond à la modélisation de l'utilisation de la machine 4. Il s'agit ici de minimiser la somme des temps pour produire les différentes pièces. Nous utilisons, pour résoudre ce problème, les contraintes exposées dans la première partie de l'analyse.\\

La 1\iere{} solution de base trouvée en utilisant les contraintes de l'entreprise est la fabrication de :\newline
\begin{itemize}
\item[\textbullet] 5 produits A
\item[\textbullet] 5 produits B
\item[\textbullet] 0 produits C
\item[\textbullet] 0 produits D
\item[\textbullet] 0 produits E
\item[\textbullet] 0 produits F\newline
\end{itemize}
Ce qui correspond à une utilisation de 60 minutes de la machine 4.\newline

Cette solution est logique car tout les produits utilisent la machine 4 donc plus la production est faible, moins cette machine est utilisée.\\

Cette solution n'étant pas réaliste, nous avons décidé d'intégrer une contrainte supplémentaire sur le nombre de produits à fabriquer. Nous souhaitons que la somme des produits fabriqués soit supérieure à 300 pièces à réaliser, soit environ 80\% de la production maximale de pièces.
Ceci se modélise par l'ajout de la contrainte suivante :\\

$$n_A + n_B + n_C + n_D + n_E + n_F \geq 300$$
d'où $$-n_A -n_B -n_C -n_D -n_E -n_F \leq -300$$

Nous obtenons donc un total de :\newline
\begin{itemize}
\item[\textbullet] 295 produits A
\item[\textbullet] 5 produits B
\item[\textbullet] 0 produits C
\item[\textbullet] 0 produits D
\item[\textbullet] 0 produits E
\item[\textbullet] 0 produits F\newline
\end{itemize}
Cela correspond à une utilisation de 638 minutes de la machine 4.\newline

Cette solution est logique car le produit nécessitant le moins la machine est le produit A (2 minutes). Elle semble également plus réaliste car elle respecte au mieux la demande du responsable personnel.


%%%%%%%%%%%%%%%%%%%%%%%%%%%%%%%%%%%%%%%%
%%%%%% Partie PROG MULTICRITERE %%%%%%%%
%%%%%%%%%%%%%%%%%%%%%%%%%%%%%%%%%%%%%%%%
\newpage
\part{Programmation linéaire multicritère}

Dans cette partie, on considère le point de vue du responsable d'entreprise. On veut construire une solution multicritère satisfaisant au mieux l'ensemble des critères. Pour cela, on va déterminer le \emph{point de mire} et la \emph{matrice des gains}. Nous nous ramènerons à un problème de programmation linéaire afin de construire une \textbf{solution de compromis}.

\section{Détermination du point de mire}
On cherche le point de mire, c'est à dire la situation idéale (non-réaliste) où tous les critères seraient parfaitement satisfaits.

On prend donc les extrema de chaque critère : maximum si le critère est à maximiser, minimum dans le cas inverse.
Les extrema  ont été calculés en partie 1 et sont rappelés ici :
\begin{itemize}
\item Comptable : Bénéfice de 10410 unités maximum
\item Atelier : Production de 376 produits maximum
\item Stocks : Stock de 1355 unités minimum
\item Personnel : Temps d'utilisation de la machine 4 de 638 minutes minimum
\end{itemize}

L'objectif du responsable commercial n'apparait pas. En effet, nous avons considéré, en accord avec le client, que son critère était de maximiser la production et que l'équilibre de produits A et B devenait une contrainte. Ainsi, son critère était redondant avec celui du comptable.

Maintenant que le point de mire est défini, nous allons chercher une solution optimale, qui soit incluse dans les domaines de critères et qui s'approche au maximum du point de mire.

\section{Matrice des gains}
\label{matgains}


\section{Recherche de la solution de compromis}

Pour déterminer une solution de compromis, on va se ramener à un problème monocritère. L'objectif le plus important sera considéré comme notre critère, et les autres seront considérés comme des contraintes. Tant qu'un compromis n'est pas trouvé, on dégradera progressivement des contraintes.

Nous avons décidé que l'objectif prépondérant était celui du comptable : maximiser les bénéfices. Les autres critères sont considérés comme des contraintes, représentées par les formules suivantes :
\todo{Ajouter formules (voir code Matlab)}

Avec la matrice des gains (ci-dessus, page \pageref{matgains}), on peut observer ...
\todo{terminer cette partie}


%%%%%%%%%%%%%%%%%%%%%%%%%%%%%%%%%%%%%%%%%%%
%%%%%% Partie ANALYSE MULTICRITERE %%%%%%%%
%%%%%%%%%%%%%%%%%%%%%%%%%%%%%%%%%%%%%%%%%%%
\newpage
\part{Analyse multicritère}

%%%%%%%%%%%%%%%%%%%%%%%%%%%%%%%%%%%%%%%%%%%
%%%%%% Partie ANNEXE %%%%%%%%%%%%%%%%%%%%%%
%%%%%%%%%%%%%%%%%%%%%%%%%%%%%%%%%%%%%%%%%%%

\newpage
\part*{Annexes}
\appendix

%%%%%%%%%%% Annexe A
\section{Calcul des coefficients pour la modélisation des objectifs du comptable}
\label{annexe}

Le but du comptable est de maximiser le bénéfice global. Pour cela, on doit
calculer le bénéfice à la vente pour chaque produit.

Le bénéfice à la vente correspond au prix de vente du produit fini auquel
on soustrait le coût de fabrication. Le coût de fabrication est composé du
coût en matières premières et coût en utilisation machines.

Pour le coût en utilisation machine, on doit penser à convertir coût horaire
des machines en coût par minute, puisque les temps unitaires d'usinage sont
exprimés en minutes.

Les calculs sont détaillés ci-après :

%Revoir comment écrire les symboles multiplication
On exprime les coûts par minute :
$$ CoutMachine_A = \frac{TempsUsinageMachine_1*CoutHoraireMachine_1 + ...}{60} $$
$$ CoutMachine_A = \frac{8*2+7*2+8*1+2*1+5*2+5*3+5*1}{60} $$
$$ CoutMachine_A = 1,167 $$

On en déduit le bénéfice :
$$ Benefice_A = PrixVente_A - CoutMachine_A - CoutMP_A $$
$$ Benefice_A = 20 - 1,167 - 13 $$
$$ Benefice_A = 5,833 $$

La calcul est analogue pour les autres produits.


%%%%%%%%%%% Annexe B
\newpage
\section{Code source Matlab}

\end{document}